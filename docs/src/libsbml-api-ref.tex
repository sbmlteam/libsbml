\documentclass{cekmanual}

\begin{document}

%=============================================================================
% Title page
%=============================================================================

\title{\textsc{libsbml} API Reference Manual}

\author{Ben Bornstein}

\authoremail{bornstei@cds.caltech.edu}

\address{Systems Biology Workbench Development Group\\
  ERATO Kitano Symbiotic Systems Project\\
  Control and Dynamical Systems, MC 107-81\\
  California Institute of Technology, Pasadena, CA 91125, USA\\[3pt]
  {\url{http://www.sbml.org/}}}

\acknowledge{Principal Investigators: John Doyle and Hiroaki Kitano}

\date{DRAFT\\[5pt]
  \today{}}

\maketitlepage


%=============================================================================
\section{Introduction}
\label{sec:quickstart-unix}
%=============================================================================

This manual is a reference for the \textsc{libsbml} application programming
interface (API).  \textsc{libsbml} is a C API for reading, writing and
manipulating the Systems Biology Markup
Language~\citep[SBML;][]{hucka:2003,finney:2003c}.  Currently, the library
supports SBML Level~1 Version~1 and Version~2, and nearly all of SBML Level~2
Version~1.  (The still-unimplemented parts of Level~2 are: support for RDF,
and support for MathML's \texttt{semantics}, \texttt{annotation} and
\texttt{annotation-xml} elements.  These will be implemented in the near
future.)  For more information about SBML, please see the references
or visit \url{http://www.sbml.org/} on the Internet.

\textsc{Libsbml} is entirely open-source and all specifications and source
code are freely and publicly available.  This document explains the library
API in detail, but does not provide general information about
\textsc{libsbml}, its use or its installation.  For that, please consult
the \emph{\textsc{libsbml} Developer's Manual}~\citep{bornstein:2003}.


%=============================================================================
\section{API Reference}
%=============================================================================

\include{api/AlgebraicRule}
\include{api/AssignmentRule}
\include{api/ASTNode}
\include{api/Compartment}
\include{api/CompartmentVolumeRule}
\include{api/EventAssignment}
\include{api/Event}
\include{api/FormulaFormatter}
\include{api/FormulaParser}
\include{api/FormulaTokenizer}
\include{api/FunctionDefinition}
\include{api/KineticLaw}
\include{api/ListOf}
\include{api/List}
\include{api/MathMLDocument}
\include{api/MathMLReader}
\include{api/memory}
\include{api/Model}
\include{api/ModifierSpeciesReference}
\include{api/ParameterRule}
\include{api/Parameter}
\include{api/ParseMessage}
\include{api/RateRule}
\include{api/Reaction}
\include{api/Rule}
\include{api/RuleType}
\include{api/SBase}
\include{api/SBMLDocument}
\include{api/SBMLReader}
\include{api/SBMLTypeCodes}
\include{api/SBMLTypes}
\include{api/SBMLWriter}
\include{api/SimpleSpeciesReference}
\include{api/SpeciesConcentrationRule}
\include{api/SpeciesReference}
\include{api/Species}
\include{api/Stack}
\include{api/StringBuffer}
\include{api/UnitDefinition}
\include{api/UnitKind}
\include{api/Unit}
\include{api/util}


%=============================================================================
% References
%=============================================================================

\bibliographystyle{apalike}
\bibliography{strings,a,b,c,d,e,f,g,h,i,j,k,l,m,n,o,p,q,r,s,t,u,v,w,x,y,z}

%=============================================================================
% The end.
%=============================================================================

\end{document}
